% Options for packages loaded elsewhere
\PassOptionsToPackage{unicode}{hyperref}
\PassOptionsToPackage{hyphens}{url}
%
\documentclass[
]{article}
\usepackage{amsmath,amssymb}
\usepackage{lmodern}
\usepackage{ifxetex,ifluatex}
\ifnum 0\ifxetex 1\fi\ifluatex 1\fi=0 % if pdftex
  \usepackage[T1]{fontenc}
  \usepackage[utf8]{inputenc}
  \usepackage{textcomp} % provide euro and other symbols
\else % if luatex or xetex
  \usepackage{unicode-math}
  \defaultfontfeatures{Scale=MatchLowercase}
  \defaultfontfeatures[\rmfamily]{Ligatures=TeX,Scale=1}
\fi
% Use upquote if available, for straight quotes in verbatim environments
\IfFileExists{upquote.sty}{\usepackage{upquote}}{}
\IfFileExists{microtype.sty}{% use microtype if available
  \usepackage[]{microtype}
  \UseMicrotypeSet[protrusion]{basicmath} % disable protrusion for tt fonts
}{}
\makeatletter
\@ifundefined{KOMAClassName}{% if non-KOMA class
  \IfFileExists{parskip.sty}{%
    \usepackage{parskip}
  }{% else
    \setlength{\parindent}{0pt}
    \setlength{\parskip}{6pt plus 2pt minus 1pt}}
}{% if KOMA class
  \KOMAoptions{parskip=half}}
\makeatother
\usepackage{xcolor}
\IfFileExists{xurl.sty}{\usepackage{xurl}}{} % add URL line breaks if available
\IfFileExists{bookmark.sty}{\usepackage{bookmark}}{\usepackage{hyperref}}
\hypersetup{
  pdftitle={Week 3 challenge: More realistid immigration sim},
  pdfauthor={Stephen Proulx},
  hidelinks,
  pdfcreator={LaTeX via pandoc}}
\urlstyle{same} % disable monospaced font for URLs
\usepackage[margin=1in]{geometry}
\usepackage{graphicx}
\makeatletter
\def\maxwidth{\ifdim\Gin@nat@width>\linewidth\linewidth\else\Gin@nat@width\fi}
\def\maxheight{\ifdim\Gin@nat@height>\textheight\textheight\else\Gin@nat@height\fi}
\makeatother
% Scale images if necessary, so that they will not overflow the page
% margins by default, and it is still possible to overwrite the defaults
% using explicit options in \includegraphics[width, height, ...]{}
\setkeys{Gin}{width=\maxwidth,height=\maxheight,keepaspectratio}
% Set default figure placement to htbp
\makeatletter
\def\fps@figure{htbp}
\makeatother
\setlength{\emergencystretch}{3em} % prevent overfull lines
\providecommand{\tightlist}{%
  \setlength{\itemsep}{0pt}\setlength{\parskip}{0pt}}
\setcounter{secnumdepth}{-\maxdimen} % remove section numbering
\ifluatex
  \usepackage{selnolig}  % disable illegal ligatures
\fi

\title{Week 3 challenge: More realistid immigration sim}
\author{Stephen Proulx}
\date{10/07/2021}

\begin{document}
\maketitle

\hypertarget{eemb-595te-fall-2021}{%
\section{EEMB 595TE Fall 2021}\label{eemb-595te-fall-2021}}

\hypertarget{programming-challenge-week-3}{%
\subsection{Programming challenge, Week
3}\label{programming-challenge-week-3}}

Following up from last week, we will talk about how to incorporate more
ecological realism into the population model. In the prior model, either
more individuals came in than left, or more left than came in, so the
population size would tend to go to 0 or to infinity. The model assumed
that the probability that a single individual came in or left was
constant in time and independent of population size, and that only a
single individual moved in/out during a time step.

Some factors that you might want to consider * Each individual has a
constant per capita probability of leaving, but only one can leave per
time step. so \(\beta(n) = \beta_0 * n\) with the constraint that
\(\beta(n) <= 1\). * Each individual has a constant probability of
leaving per time and more than one can leave per time step. So if there
are \(n\) individuals, the number who leave would be
\(y_{out} \sim \mathrm{Binomial}(n,\beta)\) * More than one individual
can come in, so \$y\_\{in\} \sim \mathrm{Poisson}(\alpha) \$. * The
resource level in the deme might affect how many individuals leave, so
\(\beta(n)\) could be an increasing function. It would need to remain
bounded between 0 and 1 * The number coming in might depend on the
attractiveness of the deme based on population size, so \(\alpha(n)\)
could be a decreasing function of \(n\). * The rate of dispersal might
be seasonal.

I would suggest thinking of an organism, and then brainstorming ideas
about what features would be important to consider.

Useful R commands

\begin{itemize}
\tightlist
\item
  \texttt{rbinom(n,size,prob)} returns \(n\) random numbers drawn from a
  binomial distribution where \(size\) individual trials each have
  probability \(prob\)
\item
  \texttt{rpois(n,lambda)} returns \(n\) random integers drawn from a
  Poisson distribution with parameter \(lambda\)
\end{itemize}

\end{document}
